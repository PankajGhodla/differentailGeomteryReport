\documentclass{article}
\usepackage[utf8]{inputenc}
\usepackage[margin=1.2in]{geometry}
\usepackage{amsmath,amssymb}
\usepackage{graphicx}
\usepackage{esvect}
\usepackage{fancyhdr}
\usepackage{amsthm}
\usepackage{tikz}
\usepackage{color}   %May be necessary if you want to colour links
\usepackage{hyperref}
\usepackage{lastpage}

\hypersetup{
    colorlinks=true, %set true if you want colored links
    linktoc=all,     %set to all if you want both sections and subsections linked
    linkcolor=blue,  %choose some color if you want links to stand out
}
% \pagestyle{plain}
% \pagestyle{fancy}
% % \fancyhf{}
% \cfoot{}
% \fancyfoot[R]{Page \thepage\ of }

\theoremstyle{plain}
\newtheorem{theorem}{Theorem}
\newtheorem{proposition}{Proposition}


\theoremstyle{definition}
\newtheorem{definition}{Definition}

\theoremstyle{remark}
\newtheorem{rem}{Remark}



% Custom Commands 
\newcommand{\R}{\mathbb{R}}


\title{Differential Geometry of Surfaces}
\author{Pankaj Ghodla }
% \date{January 2021}

\begin{document}
\maketitle
\tableofcontents

\newpage


\section{Introduction}
In this section, we will introduce some of the basic definition of curves and surfaces.
\subsection{Curves}
Intuitively, A curves can be thought as the trace of a moving particle in the space. Mathematical, a curves is defined to be the image of a function, \( \gamma: U \rightarrow \R^n \), where U \( \subset \R \)..

\begin{definition}[Parametrised curve]
    A \textbf{parametrised curve} in \( \R^n \) is a smooth function \( \gamma: U \rightarrow \R^n \), where U \( \subset \R \).
\end{definition}
Throughout, this report we will assume that smoothness mean \( \text{C}^\infty \), i.e. the function is differentiable infinitely many times.

\begin{definition}[Regular curve]
    Let \( \gamma: U \rightarrow \R^n \) be a curve. It is called regular if its derivative is non-vanishing, i.e. \( \left\lVert  \gamma^\prime(t) \right\rVert \neq 0 \), \( \forall \in U \).
\end{definition}

There are many different ways to parametrise a curve, e.g. \( \gamma(t) = (t, t^2)\) and \( \tilde{\gamma}(t) = (t^2, t^4)\). However, only one of these curve is regular, which is \( \gamma(t) \). Moreover, there are many different ways to parametrise a curves such that all the parametrisations are regular.

\begin{definition}[Unit speed curve]
    Let \( \gamma: U \rightarrow \R^n \) be a curve. It is called unit-speed, if \( \left\lVert  \gamma^\prime(t) \right\rVert = 1 \), \( \forall \in U \).
\end{definition}
We will see later on that a lot of the formulas and results relating to curves take on a much simpler form when the curve is unit-speed, e.g. curvature of a unit-speed curve, see definition \ref{definition: Curvature of a curve}, is just the norm of it's second derivative.

\begin{proposition}
    A parametrised curve is unit-speed if and only if it is regular.
\end{proposition}
{\color{red} proof ??}

{\color{red} Explain in more detail why curvature is defined in the following manner: }
\begin{definition}[Curvature of a curve] \label{definition: Curvature of a curve}
    Let \( \gamma: U \rightarrow \R^n \) be a unit-speed curve. The curvature at point \( \gamma(t) \) is defined as \[ \kappa(t) = \left\lVert \gamma^{\prime\prime} \right\rVert  \]
\end{definition}

These are all the definitions and results about curves that we need to know to understand this report.
\subsection{Surfaces}
Intuitively, a surface is a subset of \( \R^3 \) that looks like a \( \R^2 \)  in the neighbourhood of any given point, e.g. the surface of the Earth is spherical; however, it appear to be a flat plane(\( \R^2 \)) to an observer on the surface. 

\begin{definition}[Diffeomorphism]
    if \( f: U \rightarrow W \) is continuous, bijective, and smooth, and if its inverse maps \( f^-1: W \rightarrow U\) is also continuous and smooth, then f is called a diffeomorphism and \(U\) and \(W\) are called diffeomorphic.
\end{definition}

\begin{definition}[Regular Surface]
    A subset of \( \R^3 \) is a regular surface, if every point P \( \in S \), there exists a open set \( U \text{ in } \R^2\) and an open set \( W \text{ in } \R^3\) containing P such that \( S \cap W\) is diffeomorphic to \(U\). 
\end{definition}
Therefore, a surface is collection of diffeomorphisms, \( \sigma: U \rightarrow S \cap W \), which we call regular surface patches. 

\begin{definition}[Reparametrisation of surface patches]
    Let \( \sigma: U \rightarrow S\) and \( \tilde{\sigma}: \tilde{U} \rightarrow S\) be surface patches for a surface S, then \( \tilde{\sigma} \) is called a reparametrisation of \( \sigma \) if there exists a map, \( \Phi: \tilde{U} \rightarrow U \), which is smooth and bijective with smooth inverse, \( \Phi^{-1}: U \rightarrow \tilde{U} \).
\end{definition}

\begin{definition}[Tangent space]
    Let \(S\) be a regular surface. The \textbf{tangent plane} to \(S\) at the point \( p \in S\) is the set of all initial velocity vectors of regular curves in \(S\) with initial position \(p\), i.e \[ T_pS = \{ \gamma^\prime(0) | \gamma \text{ is a regular curve in S with }\gamma(0) = p\} \]
\end{definition}

These are all the definitions about surfaces that we need to understand this report.

\section{First Fundamental Form}
In this section, we will define the one of the most important object that lets us compute lengths, angles and areas on surface. It is called the \textbf{first fundamental form}. 

\begin{definition}[The fundamental form]
    The \textbf{first fundamental form} is the restriction of the inner product of the ambient space(\(\R^3\)) to the tangent space(\( T_pS\)) at point \( p \in S\).
\end{definition}

\subsection{The first fundamental form in local coordinates}


\end{document}